\documentclass{article}
\usepackage{graphicx} % Required for inserting images
\usepackage{geometry}
\usepackage{circuitikz}
\usepackage{siunitx}
\usepackage{CJKutf8}
\geometry{top=5mm, left=30mm, a4paper}




\title{Operational Amplifier Circuits Prelab}
\author{梁程捷 (B11901136), 吳奕娃 (B11901080)}
\date{}


\begin{document}
\begin{CJK*}{UTF8}{bkai}
\maketitle
%=========Inverting OP-Amp Circuit=================
\section{Inverting OP-Amp Circuit}
\textbf{Aim: Measure the output waveforms of  Inverting OP-Amp circuits in different frequencies with sinusoidal inputs.}\\
$v_{i} = 2 \sin(2\pi ft)$, \hspace{2mm}$f = 1$ \unit{\kilo\hertz},\hspace{2mm} $R_1 = 1$ \unit{\kilo\ohm},\hspace{2mm} $R_2 = 2$ \unit{\kilo\ohm}


%=======Circuit======
\begin{figure}[h]
    \centering
    \begin{circuitikz}[american]
    \draw (0,0) node[op amp] (amp) {}; 
    \draw (amp.-) -- ++(-1,0) coordinate(a) to  [R, -,l=$R_1$] ++(-3,0) to [V, v=$V_i$] ++(0,-2) node[ground]{};
    \draw (a) ++(-4,0) to [open, v=$CH1$] ++(0,-2) node[ground]{};
    \draw (amp.+) -- ++(-1,0) to  [short, -] ++(0,-1.5) node[ground]{};
    \draw (a) to [short, *-] ++(0,2) to  [R, -, l=$R_2$] ++(4,0) -- ++(0,-2.5)  coordinate(b) -- (amp.out)
    (b) to [short, *-o] ++(1,0) node[anchor = south]{$V_o$} to [open, v=$CH2$] ++(0,-2) node[ground]{};
    \draw [->](0, 0.5) to ++(0,1) node[anchor = south]{$+15V$};
    \draw [->](0, -0.5) to ++(0,-1) node[anchor = north]{$-15V$};
    \end{circuitikz}
\end{figure}

\textbf{Procedure:}\\
1. For $f = 1$ \unit{\kilo\hertz}, measure the input (CH 1) and output (CH 2) waveforms.\\
2. Repeat the measurement again in $f = 500 \sim 500$ \unit{\kilo\hertz} and make the magnitude Bode plot.\\
3. From the result, observe whether a phase shift occur in the circuit.


%=========Non-Inverting OP-Amp Circuit=================
\section{Non-Inverting OP-Amp Circuit}
\textbf{Aim: Measure the output waveforms of  Non-inverting OP-Amp circuits in different frequencies with sinusoidal inputs.}\\
$v_{i} = 2 \sin(2\pi ft)$, \hspace{2mm}$f = 1$ \unit{\kilo\hertz},\hspace{2mm} $R_1 = 1$ \unit{\kilo\ohm},\hspace{2mm} $R_2 = 2$ \unit{\kilo\ohm}


%=======Circuit======
\begin{figure}[h]
    \centering
    \begin{circuitikz}[american]
    \draw (0,0) node[op amp] (amp) {}; 
    \draw (amp.-) -- ++(-1,0) coordinate(a) to  [R, -,l=$R_1$] ++(-3,0) to [short, -] ++(0,-1.5) node[ground]{};
    \draw (amp.+) -- ++(-1,0) to [V, v=$V_i$] ++(0,-2) node[ground]{};
    \draw (amp.+) ++(-2,0) to [open, v=$CH1$] ++(0,-2) node[ground]{};
    \draw (a) to [short, *-] ++(0,2) to  [R, -, l=$R_2$] ++(4,0) -- ++(0,-2.5)  coordinate(b) -- (amp.out)
    (b) to [short, *-o] ++(1,0) node[anchor = south]{$V_o$} to [open, v=$CH2$] ++(0,-2) node[ground]{};
    \draw [->](0, 0.5) to ++(0,1) node[anchor = south]{$+15V$};
    \draw [->](0, -0.5) to ++(0,-1) node[anchor = north]{$-15V$}; 
    \end{circuitikz}
\end{figure}

\textbf{Procedure:}\\
1. For $f = 1$ \unit{\kilo\hertz}, measure the input (CH 1) and output (CH 2) waveforms.\\
2. Repeat the measurement again in $f = 500 \sim 500$ \unit{\kilo\hertz} and make the magnitude Bode plot.\\
3. From the result, observe whether a phase shift occur in the circuit.

\end{CJK*}
\end{document}
