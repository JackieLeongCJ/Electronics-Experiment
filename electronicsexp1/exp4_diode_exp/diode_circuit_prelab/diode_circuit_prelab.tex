\documentclass{article}
\usepackage{graphicx} % Required for inserting images
\usepackage{geometry}
\usepackage{circuitikz}
\usepackage{siunitx}
\usepackage{CJKutf8}
\usepackage{amsmath}
\usepackage{amssymb}
\usepackage{caption}
\usepackage{float}
\usepackage{subcaption}
\geometry{top=5mm, left=30mm, a4paper}

\title{Diode Circuits Prelab}
\author{梁程捷 (B11901136), 吳奕娃 (B11901080)}
\date{}


\begin{document}
\begin{CJK*}{UTF8}{bkai}

\maketitle
%===========Diode Circuit================
\section{Diode Circuits}
\textbf{Aim: Measure the output waveforms of Diode circuits  with sinusoidal inputs.}\\
$v_{s} = 6 \sin(2\pi ft)$ for Silicon, \hspace{2mm} $v_{s} = 12 \sin(2\pi ft)$ for Zener, \hspace{2mm} $R_L = 10$ \unit{\kilo\ohm}


%=======Circuit======
\begin{figure}[h]
    \centering
    \begin{circuitikz}[american]
    \draw [-latex] (1,4.5) -- (3,4.5);
    \draw (2, 4.5) node[above] {$i_D$};
    \draw
(0,4)   to [D, o-*, v=$v_D$]                (4,4)
        to [R, *-*, l=$R_L$,v=$v_O$(CH 2)] (4,0)
        to [short, -o]                 (0,0)
(4,4)   to [short, -o]               ++(1,0)
(4,0)   to [short, -o]               ++(1,0)
(0,4)   to [open, v=$v_S$(CH 1)]    (0,0);   
    \end{circuitikz}
\end{figure}

\textbf{Procedure:}\\
1. For $f = 1$ \unit{\kilo\hertz}, measure the input (CH 1) and output (CH 2) waveforms \textbf{in Y-t mode and X-Y mode} respectively.\\
2. For $f = 1$ \unit{\kilo\hertz}, estimate the cut-in voltage($v_D$) of the Si-based diode. \\
3. For $f = 200$ \unit{\kilo\hertz}, measure the input (CH 1) and output (CH 2) waveforms \textbf{X-Y mode}.\\
4. For $f = 1$ \unit{\kilo\hertz}, replace the Si diode with the \textbf{Zener Diode}, and achieve its \textbf{i-v} curve. \\

\textbf{*Precaution*}\\
1. Change the \textbf{coupling mode} from \textbf{ac to dc} for \textbf{CH 2}. \\
2. Check the \textbf{probe scale} of \textbf{CH 1 and CH 2} both at \textbf{1x}. \\


%=========Effects of a capacitor on the rectifier=======
\section{Effects of a Capacitor on the Rectifier}
\textbf{Aim: Measure the output waveforms of Rectifier circuit with a Capacitor with sinusoidal inputs.}\\

$v_I = 6 \sin(2\pi ft)$, \hspace{2mm} $f = 60$ \unit{\hertz}, \\
Case 1. $R = 10$ \unit{\kilo\ohm}, \hspace{2mm} $C=0.1$ \unit{\micro\farad} (104) \\
Case 2. $R = 100$ \unit{\kilo\ohm}, \hspace{1mm} $C=0.1$ \unit{\micro\farad} (104) \\
Case 3. $R = 1$ \unit{\mega\ohm}, \hspace{2mm} $C=0.2$ \unit{\micro\farad} (104 $||$ 104) \\

%=======Circuit======
\begin{figure}[h]
    \centering
    \begin{circuitikz}[american]
    \draw
(0,4)   to [D, o-*, l=D]    (3,4)
        to [C, -, l=C]    (3,0)
        to [short, -o]      (0,0)
(0,4)   to [V, v=$V_I$]     (0,0)
(-1,4)  to [open, v=$CH1$]  ++(0,-4)
(3,4)   to [short, *-*]     (5,4)
        to [R, -, l_=R]    (5,0)
        to [short, *-*]     (3,0) 
(5,4)   to [short, *-o]     (6,4)
        to [open, v=$v_O(CH 2)$] (6,0)
        to [short, o-*]     (5,0); 
    \end{circuitikz}
\end{figure}

\textbf{Procedure:}\\
1. Measure the input (CH 1) and output (CH 2) waveforms \textbf{in Y-t mode}.\\
2. Estimate the \textbf{conduction interval  $\Delta$t} and the peak-to-peak \textbf{ripple voltage $V_r$}. \\
3. Make a conclusion according to the experimental results.\\
\vspace{2mm}

\textbf{*Precaution*}\\
Change the \textbf{coupling mode} from \textbf{ac to dc} for \textbf{CH 2}. \\

%======rectifier circuit=========
\section{Voltage Regulator}
\textbf{Aim: Measure the regulator effect of a zener diode.}\\
$V_i = 9$ \unit{\volt} (DC), \hspace{2mm} $R_1 = 10$ \unit{\kilo\ohm},\hspace{2mm} $V_z < 5$ \unit{\volt}

%======circuit=============
\begin{figure}[h]
    \centering
    \begin{circuitikz}[american]
    \draw
(0,4)   to [open, v=$V_{i}$]     (0,0)
        to [short, o-]           (3,0)
        to [zD, -, l=$V_z$]    (3,4)
        to [R, -o, l=$R_1$]      (0,4)
(3,4)   to [short, *-*]     (5,4)
        to [vR, -, l_=$VR$]    (5,0)
        to [short, *-*]     (3,0) 
(5,4)   to [short, *-o]     (7,4)
        to [open, v=$V_O$]  (7,0)
        to [short, o-*]     (5,0); 
    \draw [->](0.5, 4.5) to ++(1,0) node[anchor = south east]{$I_1$};
    \draw [->](3.5, 3.5) to ++(0,-1) node[anchor = south west]{$I_Z$};
    \draw [->](5.5, 3.5) to ++(0,-1) node[anchor = south west]{$I_2$};
    \end{circuitikz}
\end{figure}

\textbf{Procedure:}\\
1. For VR = 2 \unit{\kilo\ohm}, 20 \unit{\kilo\ohm}, measure $I_1$, $I_2$, $I_z$ and $V_0$.\\
2. For VR =  400 \unit{\kilo\ohm},  600 \unit{\kilo\ohm},  800 \unit{\kilo\ohm},  1 \unit{\mega\ohm}, measure $V_o$, select one among these to measure $I_1$, $I_2$ and $I_z$\\
3. Make a conclusion according to the experimental results.


\end{CJK*}
\end{document}
